\documentclass{article}
\usepackage[utf8]{inputenc}
\usepackage[T1]{fontenc}
\usepackage{hyperref}
\usepackage{geometry}
\geometry{margin=1in}
\hypersetup{colorlinks=true, urlcolor=blue}

\begin{document}

\begin{center}
\textbf{Alberto Hern\'andez Espinosa} \\
Mexico City, Mexico $\bullet$ +52 552 383 4438 $\bullet$ albertohernandezespinosa@gmail.com \\
\url{linkedin.com/in/alberto-hernandeze} $\bullet$ \url{github.com/albertoHdzE}
\end{center}

\section{Professional Summary}
\begin{itemize}
\item AI researcher with over 10 years of experience leading scientific software development for universities and industry, specializing in artificial intelligence, cognitive science, and complexity science.
\item Expertise in designing and deploying machine learning, deep learning, and causal AI models, with a proven track record of delivering impactful solutions (e.g., 53\% cloud cost reduction).
\item Published with Oxford University and Oxford Immune Algorithmics, contributing to advancements in artificial intelligence, integrated information theory and algorithmic complexity.
\item Skilled in remote collaboration, having worked globally for 7 years, and prepared to work with interdisciplinary AI teams.
\item AI Engineer Lead at K-Square Group, driving AI integration and fostering innovation through agentic platforms and research.
\end{itemize}

\section{Skills \& Expertise}
\begin{itemize}
\item AI/ML Technologies: Python, TensorFlow, PyTorch, Scikit-learn, Keras, LangGraph, Ollama, RAG, Agents, Small Model Tuning
\item Programming Languages: Python, R, C++, SQL, Mathematica, FastAPI, React
\item Cloud \& DevOps: GCP, AWS, Docker, FastAPI
\item Data Science: Big Data, SQL/NoSQL, ETL, Pandas, NumPy, Matplotlib
\item Domains: Artificial Intelligence, Complexity Science, Cognitive Science, Reinforcement Learning, Algorithmic Information Theory, Enhancing RAG with Complexity, Theoretical Limits on Intelligence of LLMs
\end{itemize}

\section{Professional Experience}
\subsection{AI Engineer Lead, K-Square Group (Remote)}
Nov 2024--Present
\begin{itemize}
\item Conduct and lead teams for implementation of AI-driven projects, enhancing team collaboration and project execution.
\item Spearhead research and development of AI/ML applications tailored to specific industry challenges, focusing on innovation and efficiency.
\item Design and oversee the integration of advanced AI technologies, including agentic systems and complex data processing pipelines.
\item Develop and implement strategic frameworks for rapid prototyping and proof-of-concept development, aligning with organizational goals.
\item Foster a culture of experimentation and continuous improvement, mentoring teams in leveraging AI tools effectively.
\end{itemize}

\subsection{Chief AI Scientist, MilkStraw AI, San Francisco, CA (Remote)}
Oct 2022--Sep 2024
\begin{itemize}
\item Led research of AI-driven cloud resource optimization models on AWS, achieving up to 53\% cost savings for clients.
\item Designed and deployed ML pipelines using Python, TensorFlow, and PyTorch.
\item Collaborated remotely with global teams to deliver scalable AI solutions.
\end{itemize}

\subsection{Research Director, Laxford Capital (Remote)}
Jun 2019--Jul 2022
\begin{itemize}
\item Directed AI trading initiatives, developing reinforcement learning agents for financial markets.
\item Oversaw model design and implementation in Python and PyTorch.
\item Enhanced trading performance through causal AI approaches.
\end{itemize}

\subsection{Research Collaborator, Oxford Immune Algorithmics \& Oxford University, Oxford, UK (Remote with Visits)}
2015--Nov 2024
\begin{itemize}
\item Contributed to publications on integrated information theory and algorithmic complexity.
\item Implemented models in Mathematica and Python.
\item Co-authored papers with Oxford researchers, advancing cognitive science applications.
\end{itemize}

\section{Education}
\subsection{PhD in Philosophy of Science, UNAM, Mexico City, Mexico}
2015--2019 (Specialization: Artificial Intelligence and Cognitive science)
\begin{itemize}
\item Research focused on integrated information theory (Phi).
\item CONACyT Scholarship.
\item Education in association with University of Oxford, Computer Science Faculty.
\end{itemize}

\subsection{Master in Computer Sciences, UNAM, Mexico City, Mexico}
2012--2014 (Specialization: Artificial Intelligence)
\begin{itemize}
\item Thesis on computational creativity models.
\item CONACyT Scholarship.
\end{itemize}

\section{Selected Projects}
\begin{itemize}
\item Noesis: Agentic Agnostic Platform
\begin{itemize}
\item Platform for creating AI/ML models (regression, classification, clustering, anomaly detection) using LangGraph, Ollama, local tuned models, FastAPI, and React.
\item Applied to Fraud Detection in financial and insurance institutions.
\item Code: \url{https://github.com/KSProjectX/test-xx} (restricted access).
\end{itemize}

\item KS-Onboarding: Agentic Platform for Project Control and Onboarding
\begin{itemize}
\item Deep agents for organizing projects, analyzing documentation/meetings, generating insights, and responding to questions using local/external information, OCR, and voice recognition.
\item Advanced RAG enhanced with complexity science based on high-level research (\url{https://arxiv.org/abs/2505.02581}).
\item Code: \url{https://github.com/KSProjectX/KS-onboarding} (restricted access).
\end{itemize}

\item Agents for Automation of Running Tests
\begin{itemize}
\item Agentic platform using user stories/use cases in natural language or code to dynamically create pytest test code and report coverage.
\item Code: \url{https://github.com/Sucharita8/KS-qe-platform} (restricted access).
\end{itemize}

\item Stock Prices Prediction
\begin{itemize}
\item Developed predictive and autonomous decision-making models for stock prices based on Deep Reinforcement Learning and advanced ML approaches.
\item Integrated time-series analysis and reinforcement learning.
\item Deployed on cloud platforms for real-time forecasting.
\item GitHub: \url{https://github.com/albertoHdzE/finance_1}, Detailed explanation: \url{https://sites.google.com/view/complexai/FINANCE}.
\end{itemize}

\item Tononi's Phi Implementation
\begin{itemize}
\item Implemented Phi, a measure of integrated information for consciousness, in Mathematica. An approach to Algorithmic Artificial Intelligence.
\item Published findings in peer-reviewed journals, contributing to cognitive science.
\item GitHub: \url{https://github.com/albertoHdzE/Alpha}.
\end{itemize}

\item MOSAICA: DNA Complexity Analysis
\begin{itemize}
\item Investigated DNA sequence complexity using Machine learning with Python and Mathematica.
\item Correlated algorithmic complexity with DNA curvature.
\item Results published in academic papers.
\item GitHub: \url{https://github.com/albertoHdzE/MOSAICA}.
\end{itemize}
\end{itemize}

\section{Publications}
\begin{itemize}
\item 2025: ``Neurodivergent Influenceability as a Contingent Solution to the AI Alignment Problem.'' arXiv. \url{https://arxiv.org/abs/2505.02581}
\item 2025: ``SuperARC: An Agnostic Test for Narrow, General, and Super Intelligence Based On the Principles of Recursive Compression and Algorithmic Probability.'' arXiv. \url{https://arxiv.org/abs/2503.16743}
\item 2021: ``Estimations of Integrated Information Based on Algorithmic Complexity and Dynamic Querying.'' World Scientific. \url{https://www.worldscientific.com/doi/10.1142/9789811235726_0005}
\item 2017: ``Is There Any Real Substance to the Claims for a 'New Computationalism'?'' Springer. \url{https://link.springer.com/chapter/10.1007/978-3-319-58741-7_2}
\item 2013: ``Does the Principle of Computational Equivalence Overcome the Objections Against Computationalism?'' Springer. \url{https://link.springer.com/chapter/10.1007/978-3-642-37225-4_14}
\end{itemize}

\section{Certifications}
\begin{itemize}
\item Mathematics for Machine Learning
\item AI for Finance
\item Google Cloud Specialization
\item Reinforcement Learning
\item Complexity Science
\end{itemize}
Full list of 47 certifications available here: \url{https://www.linkedin.com/in/alberto-hernandeze/details/certifications/}

\section{Languages}
\begin{itemize}
\item Spanish (mother language)
\item English (advanced proficiency)
\item German (Basic)
\item French (Basic)
\item Japanese (Basic)
\end{itemize}
Certifications available here: \url{https://www.linkedin.com/in/alberto-hernandeze/details/languages/}

\end{document}